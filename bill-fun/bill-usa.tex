\documentclass[12pt,letterpaper]{article}
\usepackage[margin=2.25in,showframe,   nomarginpar,
]{geometry}
\usepackage[
series={A},
noend,
noeledsec,
nofamiliar,
noledgroup
]{reledmac}
\usepackage{fontspec}
\makeatletter
\newcommand{\congress}[1]{\def\@congress{#1}}
\newcommand{\session}[1]{\def\@session{#1}}
\newcommand{\sordinal}[1]{\def\@sordinal{#1}}
\newcommand{\cordinal}[1]{\def\@cordinal{#1}}
  \linenumincrement{1}
  \firstlinenum{1}
\usepackage[compact]{titlesec}

\def\printcongress{\@congress}% Display title
\def\printsession{\@session}% Display title
\def\printcordinal{\@cordinal}% Display title
\def\printsordinal{\@sordinal}% Display title

\makeatother


\usepackage{relsize}
\newcommand{\fakesc}[1]{\smaller\MakeUppercase{#1}\larger}
\newfontfamily\liner{Times New Roman}

\setmainfont[ItalicFont={De Vinne Italic Text}]{De Vinne Text}
\newfontfamily\titling{ITC Cheltenham Std Bold}
\usepackage{color}
\usepackage{fontspec}
\usepackage{graphicx}
\newcommand{\acthead}[1]{\begin{tabular}{@{}ccc}
\scalebox{0.75}{\begin{tabular}{c}
%%%%%%%%%%%%%
\printcongress\fakesc{\printcordinal}\ C\fakesc{ongress}\\
\printsession\fakesc{st}\ S\fakesc{ession}\\
\hspace{1ex}\\\hspace{1ex}
\end{tabular}}
&\scalebox{1.25}{\huge\titling #1}&
{\color{white}
}
\\
%\vspace*{-1ex}
&\rule{1.25in}{1pt}\\\vspace*{2ex}&\\
&\Huge\titling\MakeUppercase{an act}&
\end{tabular}
\bigskip}
\setlength\parindent{0em}
\congress{117}\cordinal{th}
\session{1}\sordinal{st}
\usepackage{hanging}
\usepackage[breakable,skins]{tcolorbox}
\newtcolorbox{legis}[1][]{%
    breakable,
    width=\textwidth,
boxsep=16pt,left=0pt,right=-16pt,top=0pt,bottom=0pt,
    enhanced,
    colback=white,
    colframe=white,
    coltitle=white,
    #1
}
\newenvironment{law}{
\begin{legis}\setlength{\parindent}{0.25in}
\beginnumbering\pstart}{\pend\endnumbering\end{legis}}
\renewcommand{\numlabfont}{\liner\normalsize}
\usepackage{lipsum}
\usepackage{setspace}
\usepackage[pagewise]{lineno}
\usepackage{changepage}
\usepackage{listings}
\usepackage{verse}
\newcounter{seclaw}
\usepackage{ifthen}
\newcommand{\lawsec}[1]{\stepcounter{seclaw}\MakeUppercase{\scriptsize\cent\bfseries 
\ifnum\theseclaw=1%
section %
\else%
sec.
\fi%
\arabic{seclaw}. #1}}
\begin{document}
\thispagestyle{empty}
\vfill

	\acthead{S. 475}
	\newfontfamily\cent[RawFeature={+liga,+dlig}, ItalicFont={TeXGyreScholaX-Italic},ItalicFeatures={
SmallCapsFont={TeXGyreScholaX-Italic},
SmallCapsFeatures={LetterSpace= 10, RawFeature={+smcp}},
},
BoldFont={TeXGyreScholaX-Bold},
SmallCapsFont={TeXGyreScholaX-Regular},
SmallCapsFeatures={LetterSpace= 10, RawFeature={+smcp}},
]{TeXGyreScholaX-Regular}

\doublespacing
\large
\hangpara{0.25in}{1}To amend title 5, United States Code, to designate Juneteenth National Independence Day as a legal public holiday.\normalsize
\setlength{\parindent}{0.25in}


\begin{law}

\textit{Be it enacted by the Senate and House of Representatives of the United States of America in Congress assembled,}

\end{law}

\clearpage


\begin{law}
	\lawsec{SHORT TITLE}

This Act may be cited as the “Juneteenth National Independence Day Act”.

\lawsec{JUNETEENTH NATIONAL INDEPENDENCE DAY AS A LEGAL PUBLIC HOLIDAY}

Section 6103(a) of title 5, United States Code, is amended by inserting after the item relating to Memorial Day the following:


``Juneteenth National Independence Day, June 19.''


\end{law}\setlength{\parindent}{0.5in}

Passed the Senate June 15, 2021.

Attest:

\hfill\textit{Secretary}




\end{document}